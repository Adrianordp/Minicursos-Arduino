\documentclass{beamer}
\usefonttheme[onlymath]{serif}
\usepackage{ragged2e}
\justifying
\usepackage[brazil]{babel}
\usepackage[utf8]{inputenc}
\usepackage{color}
\usepackage[formats]{listings}
\usetheme{Madrid}
% Berkeley
% Boadilla
% Madrid
% Montpellier
\setbeamercolor{title}{fg=white}
\setbeamercolor{frametitle}{fg=white}
\setbeamercolor{structure}{fg=myblue}

\setbeamertemplate{frametitle}{%
    \nointerlineskip%
    \begin{beamercolorbox}[wd=\paperwidth,ht=2.0ex,dp=0.6ex]{frametitle}
        \hspace*{1ex}\insertframetitle%
    \end{beamercolorbox}%
}

\makeatletter
\renewcommand{\itemize}[1][]{%
  \beamer@ifempty{#1}{}{\def\beamer@defaultospec{#1}}%
  \ifnum \@itemdepth >2\relax\@toodeep\else
    \advance\@itemdepth\@ne
    \beamer@computepref\@itemdepth% sets \beameritemnestingprefix
    \usebeamerfont{itemize/enumerate \beameritemnestingprefix body}%
    \usebeamercolor[fg]{itemize/enumerate \beameritemnestingprefix body}%
    \usebeamertemplate{itemize/enumerate \beameritemnestingprefix body begin}%
    \list
      {\usebeamertemplate{itemize \beameritemnestingprefix item}}
      {\def\makelabel##1{%
          {%
            \hss\llap{{%
                \usebeamerfont*{itemize \beameritemnestingprefix item}%
                \usebeamercolor[fg]{itemize \beameritemnestingprefix item}##1}}%
          }%
        }%
      }
  \fi%
  \beamer@cramped%
  \justifying%
  \beamer@firstlineitemizeunskip%
}
\makeatother
\usepackage{amsmath}

%----------------------------------- Setup
\title[Funções, Temporizadores e Sensor IR]{Aula 03 - SEEL 2019 \\Minicurso de Arduino \\Funções, Temporizadores e Sensor Infravermelho}
\author[Adriano]{Adriano Rodrigues}
\date{25 de outubro de 2019}

\definecolor{mygreen}{rgb}{0.6,0.6,0}
\definecolor{mygray}{rgb}{0.5,0.5,0.5}
\definecolor{myorange}{rgb}{0.8,0.4,0}
\definecolor{myorange}{rgb}{0.8,0.4,0}
\definecolor{mywhite}{rgb}{0.98,0.98,0.98}
\definecolor{myblue}{rgb}{0,0.6,0.6}

\lstdefineformat{C}
{
  \{=\newline\string\newline\indent,%
  \}=\newline\noindent\string\newline,%
  ;=[\ ]\string\space,%
}

\lstset{%
  format = C,
  backgroundcolor=\color{mywhite},
  basicstyle=\footnotesize\ttfamily,
  breakatwhitespace=false,
  breaklines=true,
  captionpos=b,
  columns=flexible,
  commentstyle=\color{mygray},
  deletekeywords={...},
  escapeinside={\%*}{*)},
  extendedchars=true,
  frame=shadowbox,
  float = H,
  keepspaces=true,
  keywordstyle=\color{myorange},
  keywordstyle=[2]\color{mygreen},
  language=C,
  morekeywords={*,...},
  numbers=left,
  numbersep=4pt,
  numberstyle=\tiny\color{black},
  rulecolor=\color{black},
  rulesepcolor=\color{myblue},
  showspaces=false,
  showstringspaces=false,          
  showtabs=false,
  stepnumber=1,
  stringstyle=\color{myblue},
  tabsize=1,
  title=\lstname,
  emphstyle=\bfseries\color{myblue},%  style for emph={}
}    

%% language specific settings:
\lstdefinestyle{Arduino}{%
    keywords={digitalWrite,delay,Serial,begin,println,print,
    available,millis,micros,parseInt,round,digitalRead,digitalWrite,
    analogRead,analogWrite,pinMode
    },%                 define keywords
    keywords=[2]{if,else,for,while,setup,loop},
    morecomment=[l]{//},%             treat // as comments
    morecomment=[s]{/*}{*/},%         define /* ... */ comments
    morecomment=[s]{//}{.},%         define /* ... */ comments
    emph={void,int,char,float,unsigned,boolean,long,HIGH,OUTPUT,LOW},% keywords to emphasize
}

\newtoggle{InString}{}% Keep track of if we are within a string
\togglefalse{InString}% Assume not initally in string

\begin{document}

%----------------------------------- Cover
\begin{frame}
	\titlepage
\end{frame}

%----------------------------------- Authors\begin{frame}
\begin{frame}
\frametitle{Aula 03}
	\begin{itemize}
		\item Funções;
		\item Declarações const e  \#define;
		\item Temporizadores;
		\item Sensor IR:
		\begin{itemize}
		\item Calibraç\~ao do Sensor IR.
		\end{itemize}
	\end{itemize}
\end{frame}

\begin{frame}
	\frametitle{Funç\~oes}
	Funç\~oes servem basicamente para descentralizar o c\'odigo de tal forma que tarefas repetidas possam ser executadas \`a parte do c\'odigo principal.\\[5pt]
	As mesmas podem ser dos tipos: \textbf{int}, \textbf{float}, \textbf{char}... Esses indicam qual tipo de vari\'avel ser\'a retornado pela funç\~ao.\\[5pt]
	Se a função não retorna nenhum valor, então a mesma será do tipo \textbf{void}.
\end{frame}

\begin{frame}[fragile]
	\frametitle{Exercício 1 - Funções}
	\begin{lstlisting}[style=Arduino,basicstyle=\scriptsize \ttfamily]
// Funcao de piscar 2 LEDs - Adriano Rodrigues.
#define LED0  13  // Maneira alternativa de declarar constantes.
#define LED1  12  // Nao consome memoria do Arduino.
boolean E[2]; // Vetor de estados - Variavel GLOBAL.
void setup() // Dispensa comentarios.{
Serial.begin (9600); pinMode(LED0, OUTPUT); pinMode(LED1, OUTPUT);
}
void loop()
{
piscar(LED0, 0); // Pisca o LED zero.
piscar(LED1, 1); // Pisca o LEL um.
delay(500);
}

// Funcao dedicada a piscar LEDs.
void piscar(int LED, int i) // Variaveis LOCAIS.
{
E[i] = !E[i];            // Alterna o valor do estado.
digitalWrite(LED, E[i]); // Escreve no LED.
}
\end{lstlisting}
\end{frame}

\begin{frame}
	\frametitle{Funç\~oes}
	Armazenar o número da porta associada a um LED em uma variável do tipo inteiro é um grande desperdício de memória.\\[5pt]
	Existem diversas maneiras otimizadas de definirem-se essas constantes: por meio de \textbf{\#define} ou por meio de \textbf{byte const}, por exemplo.\\[5pt]
	\begin{itemize}
		\item \#define LEDpin 13 \\[5pt]
		Nesse caso interface é a responsável por buscar todas as ocorrências da palavra \textit{LEDpin} e subtituí-la pelo valor 13. Ou seja, quando o programador escreve \textit{LEDpin}, o programa entenderá como $13$.
		\item byte const LEDpin = 13; \\[5pt]
		Dessa forma, você armazena o número 13 na memória do Arduino, porém em apenas 8-bits ao invés de 16-bits. Por ser constante, o compilador acusará erro caso o programador tente eneganadamente modificar o seu valor.
	\end{itemize}
\end{frame}

\begin{frame}
Qual o melhor?
	\begin{itemize}
		\item \#define LEDpin 13 \\[5pt]
		Não gasta memória, porém o número só pode ser acessado pelo compilador.
		\item byte const LEDpin = 13; \\[5pt]
		Gasta memória, porém, caso seja necessário acessar esse valor (usando ponteiros, por exemplo), existirá um endereço o qual esse valor pode ser recuperado.
	\end{itemize}
	Nas aplicações mais básicas (como as nossas), utiliza-se normalmente o \#define. As outras estratégias são melhores aproveitadas quando existe manipulação de valores armazenados nos registradores.
\end{frame}

\begin{frame}
	\frametitle{Tarefa 1 - Multitarefa com Períodos Diferentes}
	\begin{itemize}
	\item<1-> Fazer dois LEDs piscarem. LED0 com período de 1.0 segundo, LED1 com período de 1.2 segundo.
	\item<2-> Não é tão simples, né?
	\item<3-> Vamos já aprender uma forma simples e eficiente!
	\end{itemize}
\end{frame}

\begin{frame}
	\frametitle{Temporizadores}
	Funções úteis para administrar o tempo de processamento:
	\begin{itemize}
	\item delay(\textit{tempo}); $\rightarrow$ \textit{tempo} = int. Espera sem fazer nada por \textit{tempo} ms.
	\item delayMicroseconds(tempo); $\rightarrow$ \textit{tempo} = int. Espera sem fazer nada por \textit{tempo} $\mu$s.
	\item millis(); $\rightarrow$ Retorna quanto tempo (em ms) se passou desde a última inicialização.
	\item micros(); $\rightarrow$ Retorna quanto tempo (em $\mu$s) se passou desde a última inicialização.
	\end{itemize}
\end{frame}

\begin{frame}
	\frametitle{Tarefa 0 - Relógio Simples}
	\begin{itemize}
	\item Faça a implementaç\~ao de um rel\'ogio que exibe o tempo na serial.\\[5pt] Dica: ulitizar a função \textbf{millis()}.
	\end{itemize}
\end{frame}

\begin{frame}[fragile]
	\frametitle{Tarefa 0 - Relógio Simples}
	\begin{lstlisting}[style=Arduino,basicstyle=\scriptsize \ttfamily]
// Contador de tempo simples - Adriano Rodrigues.
int tempo;

void setup()
{
Serial.begin(9600);
}

void loop()
{
Serial.println(millis());
delay(1000);
}
	\end{lstlisting}
\end{frame}

\begin{frame}
	\frametitle{Temporizadores}
	Como operar duas (ou mais) tarefas que demandam per\'iodos diferentes?\\[5pt]
	Evitar o uso de delay é a forma mais eficiente de executar diversas \textit{threads}. Vamos relembrar as funções de tempo. \\[5pt]
	\begin{itemize}
	\item delay(\textit{tempo}); $\rightarrow$ \textit{tempo} = int. Espera sem fazer nada por \textit{tempo} ms.
	\item delayMicroseconds(tempo); $\rightarrow$ \textit{tempo} = int. Espera sem fazer nada por \textit{tempo} $\mu$s.
	\item millis(); $\rightarrow$ Retorna quanto tempo (em ms) se passou desde a última inicialização.
	\item micros(); $\rightarrow$ Retorna quanto tempo (em $\mu$s) se passou desde a última inicialização.
	\end{itemize}
\end{frame}

\begin{frame}[fragile]
	\frametitle{\textbf{delay()} vs \textbf{millis()}}
\begin{lstlisting}[style=Arduino,basicstyle=\scriptsize \ttfamily]
// Codigo COM delay().
void loop()
{
// Digite aqui as instrucoes.
delay(500);
}


//Codigo SEM delay().
unsigned long timer; // Evitar erros de overflow.
void loop()
{
if (millis()-timer >= 500)
{
timer = millis(); // Atualizar o timer (NAO ESQUECER!).
//Digite aqui as instrucoes.
}
}
\end{lstlisting}
	\begin{itemize}
	\item<1-> Conseguem ver a diferença?
	\item<2-> Sem delay aproveitamos melhor nosso precioso processamento.
	\end{itemize}
\end{frame}

\begin{frame}
	\frametitle{Tarefa 1}
	\begin{itemize}
	\item Fazer dois LEDs piscarem. LED0 a cada 1 segundo, LED1 a cada 1.2 segundos.
	\end{itemize}
\end{frame}

\begin{frame}[fragile]
	\frametitle{Tarefa 1 - Acionamento de LEDs Multitask}
	\begin{lstlisting}[style=Arduino,basicstyle=\scriptsize \ttfamily]
// Acionamento de LEDs Multitask - Adriano Rodrigues.
#define LED0  13  // Maneira alternativa de declarar constantes.
#define LED1  12  // Nao consome memoria do arduino.
#define temp0 500
#define temp1 600
unsigned long timer0, timer1;
boolean E[2]; // Vetor de estados - Variavel GLOBAL.
void setup() // Dispensa comentarios.
{
Serial.begin (9600); pinMode(LED0, OUTPUT); pinMode(LED1, OUTPUT);
}
void loop()
{
if (millis()-timer0>=temp0)
{
timer0 = millis();    piscar(LED0, 0); // Pisca o LED zero.
}
if (millis()-timer1>=temp1)
{
timer1 = millis();    piscar(LED1, 1); // Pisca o LEL um.
}
}
\end{lstlisting}
\end{frame}

\begin{frame}
	\frametitle{Sensor de Infravermelho}
	\begin{center}
		\pgfimage[height=80pt]{figs/figSlh01.jpg}
		\pgfimage[height=80pt]{figs/figSlh02.jpg}
		\pgfimage[height=80pt]{figs/figSlh03.jpeg}
	\end{center}
\end{frame}

\begin{frame}
	\frametitle{Sensor de Infravermelho}
	\begin{center}
		\pgfimage[height=80pt]{figs/figDslh01.jpg}\\[20pt]
		\pgfimage[height=80pt]{figs/figDslh02.png}
	\end{center}
	Ligar o terra do receptor na porta digital!
\end{frame}

\begin{frame}[fragile]
	\frametitle{Exercício 2 - Sensor Infravermelho}
	
	\begin{lstlisting}[style=Arduino,basicstyle=\scriptsize \ttfamily]
// Infrared basico - Adriano Rodrigues.
#define temp 100
int led = 13, port_sinal = 2; // Porta do LED e do SINAL.
boolean sinal = false       ; // Registrador do SINAL.
unsigned long timer         ;

void setup()
{
pinMode(led, OUTPUT)      ; // Configura porta digital do LED.
pinMode(port_sinal, INPUT); // Configura porta digital do SINAL.
}

void loop()
{
if (millis() - timer >= temp)
{
timer = millis()               ; // Atualiza timer.
sinal = digitalRead(port_sinal); // Le o SINAL.
digitalWrite(led, sinal)       ; // Escreve na porta LED.
}
}
\end{lstlisting}
\end{frame}

\begin{frame}[fragile]
	\frametitle{Exercicio 3 - IR + analogRead + Serial}
	\begin{lstlisting}[style=Arduino]
// IR + analogRead + Serial - Adriano Rodrigues.
#define temp 100
int Leitura;
unsigned long timer;

void setup()
{
Serial.begin(9600);
}
void loop()
{
if (millis() - timer >= temp)
{
timer = millis()        ;
Leitura = analogRead(A0);
Serial.println(Leitura) ;
}
}
\end{lstlisting}
\end{frame}

\begin{frame}[fragile]
	\frametitle{Tarefa 2 - Calibragem do IR}
	\begin{itemize}
	\item Exibir o valores analógicos do infravermelho pela serial;
	\item Encontrar faixa de corte (sensibilidade) de cada sensor.
	\end{itemize}
\end{frame}

\begin{frame}[fragile]
	\frametitle{Tarefa 2 - Calibragem do IR}
	\begin{lstlisting}[style=Arduino,basicstyle=\scriptsize \ttfamily]
// Calibragem do IR - Adriano Rodrigues.
#define temp 100
int  corte, Le     ; // Corte e leitura.
char char_Le = 'B' ; // Indicador de Branco ou Preto.
unsigned long timer;
void setup()
{
Serial.begin(9600);
}
void loop()
{
if (millis() - timer >= temp)
{
timer = millis();
Le = analogRead(A0); Serial.print(Le); // Recebe e imprime leitura
// Le e imprime corte.
corte = analogRead(A2); Serial.print('\t'); Serial.print(corte);
if (Le > corte) char_Le = 'B'; // Branco se Le>corte.
else char_Le = 'P'           ; // Preto caso contrario.
Serial.print('\t'); Serial.println(char_Le);
}
}
\end{lstlisting}
\end{frame}

\begin{frame}[fragile]
	\frametitle{Tarefa 2.2 - Calibragem do IR}
	\begin{lstlisting}[style=Arduino,basicstyle=\scriptsize \ttfamily]
#define temp = 100
int  corte, Le; char char_Le = 'B'; unsigned long timer;
void setup()
{
Serial.begin(9600);
}
void loop()
{
if (millis() - timer >= temp)
{
timer = millis();
Le = analogRead(A0); corte = analogRead(A2); // Recebe leitura e corte.
if (Le > corte) char_Le = 'B'; // Branco se leitura>corte.
else char_Le = 'P'           ; // Preto caso contrario.
imprimir(Le, corte, char_Le) ; // Chama impressao.
}
}
void imprimir(int l, int c, char cl)
{
Serial.print(l); Serial.print('\t'); Serial.print(c);
Serial.print('\t'); Serial.println(cl);
}
\end{lstlisting}
\end{frame}

%----------------------------------- Discussion
\begin{frame}
	\frametitle{Obrigado pela Participação.}
	\titlepage
\end{frame}
	
\end{document}