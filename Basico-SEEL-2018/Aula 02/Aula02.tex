\documentclass{beamer}
\usefonttheme[onlymath]{serif}
\usepackage{ragged2e}
\justifying
\usepackage[brazil]{babel}
\usepackage[utf8]{inputenc}
\usepackage{color}
\usepackage[formats]{listings}
\usepackage{multicol}

\usetheme{Madrid}
% Berkeley
% Boadilla
% Madrid
% Montpellier

\setbeamercolor{title}{fg=white}
\setbeamercolor{frametitle}{fg=white}
\setbeamercolor{structure}{fg=myblue}

\makeatletter
\renewcommand{\itemize}[1][]{%
  \beamer@ifempty{#1}{}{\def\beamer@defaultospec{#1}}%
  \ifnum \@itemdepth >2\relax\@toodeep\else
    \advance\@itemdepth\@ne
    \beamer@computepref\@itemdepth% sets \beameritemnestingprefix
    \usebeamerfont{itemize/enumerate \beameritemnestingprefix body}%
    \usebeamercolor[fg]{itemize/enumerate \beameritemnestingprefix body}%
    \usebeamertemplate{itemize/enumerate \beameritemnestingprefix body begin}%
    \list
      {\usebeamertemplate{itemize \beameritemnestingprefix item}}
      {\def\makelabel##1{%
          {%
            \hss\llap{{%
                \usebeamerfont*{itemize \beameritemnestingprefix item}%
                \usebeamercolor[fg]{itemize \beameritemnestingprefix item}##1}}%
          }%
        }%
      }
  \fi%
  \beamer@cramped%
  \justifying%
  \beamer@firstlineitemizeunskip%
}
\makeatother
\usepackage{amsmath}

%----------------------------------- Setup
\title[Serial, P. Analógicas, Bibliotecas e Motores]{Aula 02 - SEEL 2019 \\Minicurso de Arduino \\Comunicação Serial, Portas Analógicas, Bibliotecas e Motores}
\author[Adriano]{Adriano Rodrigues}
\date{24 de outubro de 2019}

\definecolor{mygreen}{rgb}{0.6,0.6,0}
\definecolor{mygray}{rgb}{0.5,0.5,0.5}
\definecolor{myorange}{rgb}{0.8,0.4,0}
\definecolor{myorange}{rgb}{0.8,0.4,0}
\definecolor{mywhite}{rgb}{0.98,0.98,0.98}
\definecolor{myblue}{rgb}{0,0.6,0.6}

\lstdefineformat{C}
{
  \{=\newline\string\newline\indent,%
  \}=\newline\noindent\string,%
  ;=[\ ]\string\space,%
}

\lstset{%
  format = C,
  backgroundcolor=\color{mywhite},
  basicstyle=\footnotesize\ttfamily,
  breakatwhitespace=false,
  breaklines=true,
  captionpos=b,
  columns=flexible,
  commentstyle=\color{mygray},
  deletekeywords={...},
  escapeinside={\%*}{*)},
  extendedchars=true,
  frame=shadowbox,
  keepspaces=true,
  keywordstyle=\color{myorange},
  keywordstyle=[2]\color{mygreen},
  language=C,
  morekeywords={*,...},
  numbers=left,
  numbersep=4pt,
  numberstyle=\tiny\color{black},
  rulecolor=\color{black},
  rulesepcolor=\color{myblue},
  showspaces=false,
  showstringspaces=false,          
  showtabs=false,
  stepnumber=1,
  stringstyle=\color{myorange},
  tabsize=1,
  title=\lstname,
  emphstyle=\bfseries\color{myblue},%  style for emph={}
}    

%% language specific settings:
\lstdefinestyle{Arduino}{%
    keywords={digitalWrite,delay,Serial,begin,println,print,
    available,millis,micros,parseInt,round,digitalRead,digitalWrite,
    analogRead,analogWrite,pinMode
    },%                 define keywords
    keywords=[2]{if,else,for,while,setup,loop},
    morecomment=[l]{//},%             treat // as comments
    morecomment=[s]{/*}{*/},%         define /* ... */ comments
    morecomment=[s]{//}{.},%         define // ... . comments
    emph={void,int,float,unsigned,boolean,long,HIGH,OUTPUT,LOW},%        keywords to emphasize
}

\newtoggle{InString}{}% Keep track of if we are within a string
\togglefalse{InString}% Assume not initally in string

\begin{document}

%----------------------------------- Cover
\begin{frame}
	\titlepage
\end{frame}

%----------------------------------- Conteúdo
\section{Conteúdo}
\begin{frame}
\frametitle{Aula 02}
	\begin{itemize}
		\item Comunicação Serial;
		\item Sinais Analógicos:
		\begin{itemize}
			\item Entradas Analógicas;
			\item ``Saídas" Analógicas (PWM).
		\end{itemize}
		\item Motor CC;
		\item Bibliotecas;
		\item Servo Motor.
	\end{itemize}
\end{frame}

%---------------------------------- Índice
\section{Índice}
\begin{frame}
\frametitle{Índice}
\begin{multicols}{2}
\tableofcontents
\end{multicols}
\end{frame}

%----------------------------------- Comunicação Serial
\section{Comunicação Serial}
\begin{frame}[fragile]
	\frametitle{Comunicação Serial}
	\begin{center}
		\pgfimage[height=180pt]{figs/figSerial}
	\end{center}
\end{frame}


\begin{frame}[fragile]
	\frametitle{Comunicação Serial}
	Setup():
	\begin{itemize}
		\item Serial.begin(\textit{baudrate})
		\textit{baudrate} = int 300, 600, 1200, 2400, 4800, \textbf{9600}, 14400, 19200, 28800, 38400, 56700, 115200.
	\end{itemize}
	Loop():
	\begin{itemize}
		\item Serial.print(\textit{texto}); $\rightarrow$ \textit{texto} = string ou char.
		\item Serial.println(\textit{texto}); $\rightarrow$ Imprime e pula uma linha.
		\item Uma string pode ser escrita diretamente entre aspas (``exemplo") e um caractere entre aspas simples (`a').
		\item Serial.available();  $\rightarrow$ Retorna a quantidade de bytes a serem lidos.
		\item Serial.read(); $\rightarrow$ Retorna o byte ou caractere lido na pilha.
		\item Serial.parseInt(); $\rightarrow$ Retorna número inteiro lido na serial.
		\item atoi(\textit{string}); $\rightarrow$ Retorna um número inteiro.\\ Exemplo \textit{int} val = atoi(``12345") $\rightarrow$ val = 12345.
	\end{itemize}
\end{frame}

%----------------------------------- Exercício 01
\section{Exercício 01}
\begin{frame}[fragile]
	\frametitle{Exercício 01 - Comunicação Serial de Inteiros}
	
	\begin{lstlisting}[style=Arduino]
	// Comunicacao Serial de Inteiros - Adriano Rodrigues.
	int buf = 0;
	void setup() // Sempre utilizar Serial begin no setup.{
	Serial.begin(9600);// Configura Serial para 9600 de baud rate.}
	void loop()	{
	if (Serial.available()>0)	{
	buf = Serial.parseInt(); // Interpreta o dado lido como int.
	Serial.println(buf*2)  ; // Imprime BUF multiplicado por 2. } } 	\end{lstlisting}
\end{frame}

%----------------------------------- Analógico vs Digital
\section{Analógico vs Digital}
\begin{frame}
	\frametitle{Analógico \textit{vs} Digital}
	\begin{center}
		\pgfimage[height=180pt]{figs/figAd01.jpg}
	\end{center}
\end{frame}

\begin{frame}
	\frametitle{Analógico \textit{vs} Digital}
	\begin{center}
		\pgfimage[height=150pt]{figs/figAd02.jpeg}
	\end{center}
\end{frame}

\begin{frame}
	\frametitle{Analógico \textit{vs} Digital}
	\begin{center}
		\pgfimage[height=180pt]{figs/figAd03.jpg}
	\end{center}
\end{frame}

\begin{frame}
	\frametitle{Analógico \textit{vs} Digital}
	\begin{center}
		\pgfimage[height=130pt]{figs/figAd04.jpg}
	\end{center}
\end{frame}

%----------------------------------- Funções Analógicas pt1
\section{Funções Analógicas pt1}
\begin{frame}
	\frametitle{Funções Analógicas parte 1}
	Leitura Analógica:
	\begin{itemize}
		\item analogRead(\textit{pino});\\
		pino = int de 0 a 5, retorna de 0 a 1023.
	\end{itemize}
	Escrita Analógica:
	\begin{itemize}
		\item Infelizmente o Arduino Uno não conta com essa função. Ao menos não exatamente...
	\end{itemize}
\end{frame}

%----------------------------------- Exercício 02
\section{Exercício 02}
\begin{frame}[fragile]
	\frametitle{Exercício 02 - Frequência Modificada por Entrada Analógica}
	
	\begin{lstlisting}[style=Arduino,basicstyle=\scriptsize \ttfamily]
// Frequencia do LED + Porta analogica - Adriano Rodrigues.

int     led     = 13  ; // Porta do LED.
int     periodo = 0   ; // Valor do period.
boolean estado  = true; // Estado do led.

void setup()
{
pinMode(led, OUTPUT); // Configura a porta digital ligada ao LED.
}

void loop()
{
estado  = !estado       ; // Alterna o valor do estado.
periodo = analogRead(0) ; // Le o valor do potenciometro.
digitalWrite(led,estado);
delay(periodo/2)        ; // Periodo determinado pela leitura analogica.
}	\end{lstlisting}
\end{frame}

%----------------------------------- Potenciômetro
\section{Potenciômetro}
\begin{frame}
	\frametitle{Potenciômetro}
		\pgfimage[height=200pt]{figs/figPot} \hspace{0pt}
		\pgfimage[height=110pt]{figs/figPot2}
\end{frame}

%----------------------------------- PWM
\section{PWM}
\begin{frame}
	\frametitle{PWM - \textit{Pulse Width Modulation} - Modulação por Largura de Pulso}
	\begin{center}
		\pgfimage[height=180pt]{figs/figPwm.png}
	\end{center}
\end{frame}

\begin{frame}
	\frametitle{PWM - \textit{Pulse Width Modulation} - Modulação por Largura de Pulso}
	\begin{center}
		\pgfimage[height=180pt]{figs/figPwm2.png}
	\end{center}
\end{frame}

%----------------------------------- Funções Analógicas pt2
\section{Funções Analógicas pt2}
\begin{frame}
	\frametitle{Funções Analógicas parte 2}
	Leitura Analógica:
	\begin{itemize}
		\item analogRead(\textit{pino});\\
		pino = int de 0 a 5, retorna de 0 a 1023.
	\end{itemize}
	Escrever PWM:
	\begin{itemize}
		\item analogWrite(\textit{pino}, \textit{valor});\\
		pino = pinos digitais com '$\sim$'.
		valor =  int de 0 a 255.\\
		\textbf{É necessário definir o pino como saída!}
	\end{itemize}
\end{frame}

%----------------------------------- Exercício 03
\section{Exercício 03}
\begin{frame}[fragile]
	\frametitle{Exercício 03 - Brilho do LED via PWM \\ Com LED \textbf{EXTERNO}!}
	\begin{lstlisting}[style=Arduino,basicstyle=\scriptsize \ttfamily]
// Brilho Variavel do LED via PWM - Adriano Rodrigues.
// Porta do LED.
int led = 11; // Deve ser uma porta PWM.
int buf = 0 ;
void setup()
{
Serial.begin(9600); pinMode(led, OUTPUT);
}
void loop()
{
if (Serial.available()>0)
{
buf = Serial.parseInt(); // Interpreta o dado lido como um int.
}
analogWrite(led,buf);
delay(100);
}	\end{lstlisting}
\end{frame}

%----------------------------------- Tarefa 01
\section{Tarefa 01}
\begin{frame}
	\frametitle{Tarefa 01 - Brilho Variável do LED via PWM\\Com LED \textbf{EXTERNO}!}
	Fazer o brilho do LED aumentar com o tempo.\\[5pt]
	Quando chegar ao brilho máximo, a sequência deve ser recomeçada.
\end{frame}

\begin{frame}[fragile]
	\frametitle{Tarefa 01 - Brilho Variável do LED via PWM\\Com LED \textbf{EXTERNO}!}
	\begin{lstlisting}[style=Arduino,basicstyle=\scriptsize \ttfamily]
// Brilho Variavel do LED via PWM - Adriano Rodrigues.

// Porta do LED.
int led = 11; // Deve ser uma porta PWM.

void setup()
{
// Configuracao da porta digital do LED.
pinMode(led, OUTPUT);
}

void loop()
{
for(int i = 0; i < 255; i++)
{
analogWrite(led,i);
delay(10)         ;
}
}	\end{lstlisting}
\end{frame}

%----------------------------------- Acionamento dos Motores
\section{Acionamento dos Motores}
\begin{frame}
	\frametitle{Acionamento dos Motores}
	\begin{center}
		\pgfimage[height=100pt]{figs/figTip31.jpg} \hspace{15pt}
		\pgfimage[height=180pt]{figs/figDriver.jpg}
	\end{center}
\end{frame}

%----------------------------------- Exercício 04
\section{Exercício 04}
\begin{frame}[fragile]
	\frametitle{Exercício 04 - Motor DC}
	\begin{lstlisting}[style=Arduino,basicstyle=\scriptsize \ttfamily]
// Velocidade do Motor - Adriano Rodrigues.
int comando = 200; // Ciclo de trabalho = 200/255.
int port_motor = 3;

void setup()
{
pinMode(port_motor,OUTPUT); // Definir como saida.
}

void loop()
{
analogWrite(port_motor,comando); // Escreve em PWM.
delay(100);
}	\end{lstlisting}
\end{frame}

%----------------------------------- Tarefa 02
\section{Tarefa 02}
\begin{frame}
	\frametitle{Tarefa 02 - Motor DC com Controle Externo}
	Controlar a velocidade de um motor manualmente por meio da leitura da tensão de um potênciômetro.
\end{frame}

\begin{frame}[fragile]
	\frametitle{Tarefa 02 - Motor DC com Controle Externo}
	\begin{lstlisting}[style=Arduino,basicstyle=\scriptsize \ttfamily]
// Velocidade do Motor com Potenciometro - Adriano Rodrigues.
int regulacao, comando, port_motor = 3;

void setup()
{
pinMode(port_motor,OUTPUT); // Definir como saida.
}

void loop()
{
regulacao = analogRead(A2)     ; // Potenciometro.
comando   = regulacao / 4      ; // Mapeia 0-1023 para 0-255.
analogWrite(port_motor,comando); // Escreve em PWM.
delay(100);
}	\end{lstlisting}
\end{frame}

%----------------------------------- Servomotores
\section{Servomotores}
\begin{frame}
	\frametitle{Servomotores}
	\begin{center}
		\pgfimage[height=180pt]{figs/figServo}
	\end{center}
\end{frame}

\begin{frame}
	\frametitle{Servomotores}
	\begin{center}
		\pgfimage[height=180pt]{figs/figServoSignal.jpg}
	\end{center}
\end{frame}

%----------------------------------- Bibliotecas
\section{Bibliotecas}
\begin{frame}
	\frametitle{Bibliotecas}
	\#include$<$\textit{biblioteca.h}$>$
	\begin{itemize}
	\item \#include$<$\textit{Servo.h}$>$
	\item \#include$<$\textit{Wire.h}$>$
	\item \#include$<$\textit{Mouse.h}$>$
	\item \#include$<$\textit{Keyboard.h}$>$
	\item \#include$<$\textit{Stepper.h}$>$
	\item \#include$<$\textit{dht.h}$>$
	\end{itemize}
\end{frame}

%----------------------------------- Exercício 05
\section{Exercício 05}
\begin{frame}[fragile]
	\frametitle{Exercício 05 - Servo + Biblioteca}
	\begin{lstlisting}[style=Arduino,basicstyle=\scriptsize \ttfamily]
#include <Servo.h> // Adiciona a biblioteca de servos.
Servo servo1; // Cria uma variavel servo1 na classe Servo.
int pos = 0 ; // Posicao em graus.
void setup()
{
servo1.attach(13); // Anexa o servo a porta 13.
}
void loop()
{
for (pos = 0; pos <= 180; pos += 1) // Incrementa o angulo.
{
servo1.write(pos); // Escreve o angulo desejado no servo.
delay(15);
}
for (pos = 180; pos >= 0; pos -= 1) // Decrementa o angulo.
{
servo1.write(pos); // Escreve o angulo desejado no servo.
delay(15);
}
}	\end{lstlisting}
\end{frame}

%----------------------------------- Tarefa 03
\section{Tarefa 03}
\begin{frame}
	\frametitle{Tarefa 03 - Controlar Ângulo do Servo por Serial}
	Controlar o ângulo de um motor por meio de comunicação serial.
\end{frame}

\begin{frame}[fragile]
	\frametitle{Tarefa 03 - Controlar Ângulo do Servo por Serial}
	\begin{lstlisting}[style=Arduino,basicstyle=\scriptsize \ttfamily]
// Controlar Angulo do Servo por Serial - Adriano Rodrigues.
#include <Servo.h> // Adiciona a biblioteca de servos.

Servo servo1; // Cria uma variavel servo1 na classe Servo.

int pos = 0 ; // Posicao em graus.

void setup()
{
Serial.begin(9600);
servo1.attach(13); // Anexa o servo a porta 13.
}
void loop()
{
if (Serial.available() > 0)
{
pos = Serial.parseInt(); // Interpreta o dado lido como um int.
}
servo1.write(pos);
delay(50);
}	\end{lstlisting}
\end{frame}

%----------------------------------- Fim!
\section{Fim!}
\begin{frame}
	\frametitle{Até a próxima aula!}
	\titlepage
\end{frame}
	
\end{document}