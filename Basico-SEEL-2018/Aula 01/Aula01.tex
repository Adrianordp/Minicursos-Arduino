\documentclass{beamer}
\usefonttheme[onlymath]{serif}
\usepackage{ragged2e}
\justifying
\usepackage[brazil]{babel}
\usepackage[utf8]{inputenc}
\usepackage{color}
\usepackage[formats]{listings}
\usepackage{multicol}

\usetheme{Madrid}
% Berkeley
% Boadilla
% Madrid (*)
% Montpellier
\setbeamercolor{title}{fg=white}
\setbeamercolor{frametitle}{fg=white}
\setbeamercolor{structure}{fg=myblue}

\setbeamertemplate{frametitle}{%
    \nointerlineskip%
    \begin{beamercolorbox}[wd=\paperwidth,ht=2.0ex,dp=0.6ex]{frametitle}
        \hspace*{1ex}\insertframetitle%
    \end{beamercolorbox}%
}

\makeatletter
\renewcommand{\itemize}[1][]{%
  \beamer@ifempty{#1}{}{\def\beamer@defaultospec{#1}}%
  \ifnum \@itemdepth >2\relax\@toodeep\else
    \advance\@itemdepth\@ne
    \beamer@computepref\@itemdepth% sets \beameritemnestingprefix
    \usebeamerfont{itemize/enumerate \beameritemnestingprefix body}%
    \usebeamercolor[fg]{itemize/enumerate \beameritemnestingprefix body}%
    \usebeamertemplate{itemize/enumerate \beameritemnestingprefix body begin}%
    \list
      {\usebeamertemplate{itemize \beameritemnestingprefix item}}
      {\def\makelabel##1{%
          {%
            \hss\llap{{%
                \usebeamerfont*{itemize \beameritemnestingprefix item}%
                \usebeamercolor[fg]{itemize \beameritemnestingprefix item}##1}}%
          }%
        }%
      }
  \fi%
  \beamer@cramped%
  \justifying%
  \beamer@firstlineitemizeunskip%
}
\makeatother
\usepackage{amsmath}

%----------------------------------- Setup
\title[Intro, P. Digitais e Emulação no Proteus]{Aula 01 - SEEL 2019\\Minicurso de Arduino \\ Introdução, Portas Digitais e Emulação no Proteus}
\author[Adriano]{Adriano Rodrigues}
\date{23 de outubro de 2019}

\definecolor{mygreen}{rgb}{0.6,0.6,0}
\definecolor{mygray}{rgb}{0.5,0.5,0.5}
\definecolor{myorange}{rgb}{0.8,0.4,0}
\definecolor{myorange}{rgb}{0.8,0.4,0}
\definecolor{mywhite}{rgb}{0.98,0.98,0.98}
\definecolor{myblue}{rgb}{0,0.6,0.6}

\lstdefineformat{C}
{
  \{=\newline\string\newline\indent,%
%  \}=\newline\noindent\string\newline,%
  \}=\newline\noindent\string,%
  ;=[\ ]\string\space,%
}

\lstset{%
  format = C,
  backgroundcolor=\color{mywhite},
  basicstyle=\footnotesize\ttfamily,
  breakatwhitespace=false,
  breaklines=true,
  captionpos=b,
  columns=flexible,
  commentstyle=\color{mygray},
  deletekeywords={...},
  escapeinside={\%*}{*)},
  extendedchars=true,
  frame=shadowbox,
  keepspaces=true,
  keywordstyle=\color{myorange},
  keywordstyle=[2]\color{mygreen},
  language=C,
  morekeywords={*,...},
  numbers=left,
  numbersep=4pt,
  numberstyle=\tiny\color{black},
  rulecolor=\color{black},
  rulesepcolor=\color{myblue},
  showspaces=false,
  showstringspaces=false,          
  showtabs=false,
  stepnumber=1,
  stringstyle=\color{myorange},
  tabsize=1,
  title=\lstname,
  emphstyle=\bfseries\color{myblue},%  style for emph={}
}    

%% language specific settings:
\lstdefinestyle{Arduino}{%
    keywords={digitalWrite,delay,Serial,begin,println,print,
    available,millis,micros,parseInt,round,analogWrite,pinMode
    },%                 define keywords
    keywords=[2]{if,else,for,while,setup,loop},
    morecomment=[l]{//},%             treat // as comments
    morecomment=[s]{/*}{*/},%         define /* ... */ comments
    morecomment=[s]{//}{.},%         define // ... . comments
    emph={void,int,float,unsigned,long,HIGH,OUTPUT,INPUT,LOW},%        keywords to emphasize
}

\newtoggle{InString}{}% Keep track of if we are within a string
\togglefalse{InString}% Assume not initally in string

\begin{document}

%----------------------------------- Cover
\begin{frame}
	\titlepage
\end{frame}

%----------------------------------- Authors
\section{Conteúdo}
\begin{frame}
\frametitle{Aula 01 - Conteúdo}
	\begin{itemize}
		\item Introdução;
		\item Apresentação da Placa;
		\item Aplicações;
		\item Introdução à Interface;
		\item Introdução à Programação;
		\item Portas Digitais:
		\begin{itemize}
			\item Sa\'idas Digitais;
			\item Entradas Digitais.
		\end{itemize}
		\item Emulação do Arduino no Proteus.
	\end{itemize}
\end{frame}

%---------------------------------- Índice
\section{Índice}
\begin{frame}
\frametitle{Índice}
\begin{multicols}{2}
\tableofcontents
\end{multicols}
\end{frame}

%----------------------------------- O Arduino
\section{O Arduino}
\begin{frame}
\frametitle{O Arduino}
	\begin{itemize}
		\item Site oficial: arduino.cc;
		\item Plataforma eletrônica de código aberto baseada em hardware e software livres de fácil utilização. Foi pensado para que qualquer pessoa possa fazer seus próprios projetos interativos;
		\item Modelos: Uno, Mega (2560, ADK), Due, Leonardo etc...
	\end{itemize}
\end{frame}

%----------------------------------- O Arduino UNO R3
\section{O Arduino UNO R3}
\begin{frame}
	\frametitle{O Arduino UNO R3}
	\begin{itemize}
		\item 14 portas digitais (6 podem ser configuradas PWM);
		\item 6 entradas analógicas (10 bits);
		\item Microcontrolador ATmega328P ($16$MHz de clock);
		\item Opera em $5$V;
		\item Alimentado entre $7$V-$12$V.
	\end{itemize}
\end{frame}

\begin{frame}
	\frametitle{O Arduino UNO R3}
	\begin{center}
		\pgfimage[height=180pt]{figs/figUnoR3.jpg}
	\end{center}
\end{frame}

%----------------------------------- Exemplos
\section{Aplicações}
\begin{frame}
	\frametitle{Aplicações}
	\begin{itemize}
	\item Robótica:
	\end{itemize}
	\begin{center}
		\pgfimage[height=130pt]{figs/figRobot}
	\end{center}
\end{frame}
\begin{frame}
	\frametitle{Aplicações}
	\begin{itemize}
	\item Sensoriamento:
	\end{itemize}
	\begin{center}
		\pgfimage[height=130pt]{figs/figSonar}
	\end{center}
\end{frame}
\begin{frame}
	\frametitle{Aplicações}
	\begin{itemize}
	\item Automação:
	\end{itemize}
	\begin{center}
		\pgfimage[height=130pt]{figs/figApps}
	\end{center}
\end{frame}

%----------------------------------- Introdução à Interface
\section{Introdução à Interface}
\begin{frame}
	\frametitle{Introdução à Interface}
	\begin{center}
		\pgfimage[height=150pt]{figs/figIDE}
	\end{center}
	\begin{itemize}
		\item[1] Verificar e compilar o Código;
		\item[2] Fazer Upload do Código no Arduino;
		\item[3] Criar Novo Sketch;
		\item[4] Abrir outro Sketch;
		\item[5] Salvar este Sketch;
	\end{itemize}
\end{frame}

\begin{frame}
	\frametitle{Introdução à Interface}
	\begin{center}
		\pgfimage[height=130pt]{figs/figIDE}
	\end{center}
	\begin{itemize}
		\item[6]  Monitor Serial - Tela de monitoramento da serial;
		\item[7]  Título do Sketch;
		\item[8]  Campo do Código;
		\item[9]  Status da Interface;
		\item[10] Status e Avisos Relacionados ao Código;
		\item[11] Informação de Conexão com o Arduino.
	\end{itemize}
\end{frame}

%----------------------------------- Introdução à Programação
\section{Introdução à Programação}
\begin{frame}
	\frametitle{Introdução à Programação}
	Tipos de Vari\'aveis
	\begin{itemize}
		\item int - inteiros de -32768 at\'e 32767;
		\begin{itemize}
			\item \textit{int} x = 5;
			\item \textit{int} x = 5, y = 4, z = 10;
		\end{itemize}
		\item long - inteiros de -2.147.483.648 at\'e 2.147.483.647;
		\item float - n\'umeros reais de -3,4028235e38 at\'e 3,4028235e38;
		\item char - caracteres (exemplo: \textit{char} 'A');
		\item byte - binários de d0 até d255 (b10010 = d18);
		\item boolean - \textit{false} ou \textit{true};
		\item array:
		\begin{itemize}
			\item \textit{int} myInt[6]; (5 elementos + fim de curso)
			\item \textit{int} myPins[] = \{2,4,8,3,5\};
			\item \textit{int} mySensors[6] = \{1,4,-8,3,2\};
			\item \textit{char} message[6] = "Hello";
			\item Primeiro elemento é referido como ``myPins[0]'', por exemplo;
		\end{itemize}
	\end{itemize}
\end{frame}

\begin{frame}[fragile]
	\frametitle{Introdução à Programação}
	\begin{itemize}
		\item O B\'asico
	\end{itemize}
	
	\begin{lstlisting}[style=Arduino]
	void setup()	{
	// Esse codigo roda somente uma vez.	}
	void loop()	{
	// Codigo principal que roda recursivamente. }	\end{lstlisting}
\end{frame}

\begin{frame}[fragile]
\frametitle{Introdução à Programação}
	\begin{itemize}
		\item if / else if / else:
	\end{itemize}
	
	\begin{lstlisting}[style=Arduino]
	if (variavel < condicao1) {
	// Fazer A.	}
	else if (variavel >= condicao2)	{
	// Fazer B.	}
	else {
	// Fazer C.	}	\end{lstlisting}
\end{frame}

\begin{frame}[fragile]
\frametitle{Introdução à Programação}
	\begin{itemize}
		\item for:
	\end{itemize}
	
	\begin{lstlisting}[style=Arduino]
	for (inicializacao;condicao;incremento)	{
	// Instrucoes.	}	\end{lstlisting}
	Exemplo:
	\begin{lstlisting}[style=Arduino]
	for (int i = 0; i<= 255 ; i++)	{
	 a[i] = 2*i;	}	\end{lstlisting}
\end{frame}

\begin{frame}[fragile]
	\frametitle{Introdução à Programação}
	\begin{itemize}
		\item while:
	\end{itemize}
	
	\begin{lstlisting}[style=Arduino]
	while(condicao)	{
	// Comando.	}	\end{lstlisting}
	Exemplo:
	\begin{lstlisting}[style=Arduino]
	int var = 0;
	while(var<200)	{
	var++;	}	\end{lstlisting}
\end{frame}

%--------------------------------------------------- Funções Digitais
\section{Funções Digitais}
\begin{frame}
	\frametitle{Funções Digitais}
	No \textit{\textbf{setup()}}. Definição das portas:
	\begin{itemize}
		\item pinMode(\textit{pino},\textit{direção});\\
		pino = int de 0 a 13\\
		direção = INPUT ou OUTPUT.
	\end{itemize}
	No \textit{\textbf{loop()}}:
	\begin{itemize}
		\item digitalRead(\textit{pino});\\
		pino = int de 0 a 13\\
		Função de \textbf{leitura}! Retorna booleano.
		\item digitalWrite(\textit{pino}, \textit{estado});\\
		pino = int de 0 a 13\\
		estado = HIGH ou LOW.
	\end{itemize}
\end{frame}

%-------------------------------------------------- Exercício 01
\section{Exercício 1}
\begin{frame}
	\frametitle{Exercício 1 - Introdução à IDE - Pisca LED}
	\begin{center}
		\pgfimage[height=180pt]{figs/figIDE}
	\end{center}
	\begin{itemize}
	
		\item[8] Escrever o código aqui.
	\end{itemize}
\end{frame}

\begin{frame}[fragile]
	\frametitle{Exercício 1 - Introdução à IDE - Pisca LED}
	
	\begin{lstlisting}[style=Arduino,basicstyle=\scriptsize \ttfamily]
	// LED piscando - Adriano Rodrigues.
	
	// Porta do LED.
	int led = 13;
	
	void setup()	{
	// Configuracao da porta digital do LED.
	pinMode(led, OUTPUT);	}
	
	void loop()	{
	digitalWrite(led, HIGH); // Escreve 5V na porta LED.
	delay(2000)            ; // Espera por 2000 milissegundos.
	digitalWrite(led, LOW) ; // Escreve 0V na porta LED.
	delay(2000)            ; // Espera por 2000 milissegundos.	} \end{lstlisting}
\end{frame}

\begin{frame}
	\frametitle{Exercício 1 - Introdução à IDE - Pisca LED}
	\begin{center}
		\pgfimage[height=110pt]{figs/figIDE}
	\end{center}
	\begin{itemize}
		\item[1] Verificar Coerência; + Salvar o código:
		\begin{itemize}
			\item Browser Padrão;
			\item Não separar nomes, ou iniciar por números, ou por caracteres especiais;
			\item Esse procedimento cria uma pasta com um arquivo \textit{.ino} com o texto da sketch.
		\end{itemize}
		\item[2] Fazer Upload;
		\item[9] Barra de progresso do Upload;
		\item[10] Verificar Status do Upload;
	\end{itemize}
\end{frame}

\begin{frame}
	\frametitle{Exercício 1 - Introdução à IDE - Pisca LED}
	\begin{center}
		\pgfimage[width=\textwidth]{figs/figEx01}
	\end{center}
\end{frame}

\begin{frame}
	\frametitle{Exercício 1 - Introdução à IDE - Pisca LED}
	\begin{center}
		\pgfimage[width=0.8\textwidth]{figs/figLED}
	\end{center}
\end{frame}

%------------------------------------------------ Exercício 2.0
\section{Exercício 2}
\begin{frame}
	\frametitle{Exercício 2 - Esquemático}
	\begin{center}
		\pgfimage[width=\textwidth]{figs/figEx02}
	\end{center}
\end{frame}

\begin{frame}[fragile]
	\frametitle{Exercício 2.0 - Botão de Comando}
	
	\begin{lstlisting}[style=Arduino]
// LED com if (botao) - Adriano Rodrigues.
int led   = 13; // Porta do LED.
int botao = 12; // Porta do Botao.
void setup() {
pinMode(led, OUTPUT) ;// Configuracao da porta digital do LED.
pinMode(botao, INPUT);// Configuracao da porta digital do BOTAO.}
void loop() {
if (digitalRead(botao) == HIGH){
digitalWrite(led, HIGH);// Escreve 5V na porta LED.}
else // ou 'else if (digitalRead(botao) == LOW)'. {
digitalWrite(led, LOW); // Escreve 0V na porta LED. } } 	\end{lstlisting}
\end{frame}

%-------------------------------------------- Exercício 2.1
\begin{frame}[fragile]
	\frametitle{Exercício 2.1 - Botão de Comando - Simplificado}
	
	\begin{lstlisting}[style=Arduino]
// LED com if (botao) - Adriano Rodrigues.
int led = 13, botao = 12;// Porta do LED - Porta do botao.
void setup(){
pinMode(led, OUTPUT); pinMode(botao, INPUT);// Configurar PDs. }
void loop() {
if (digitalRead(botao) == HIGH)
digitalWrite(led, HIGH);// Escreve 5V na porta LED.
else
digitalWrite(led, LOW);// Escreve 0V na porta LED. }	\end{lstlisting}
\end{frame}

%-------------------------------------------- Exercício 2.2
\begin{frame}[fragile]
	\frametitle{Exercício 2.2 - Botão de Comando - Simplificado Menor}
	
	\begin{lstlisting}[style=Arduino]
// LED com if (botao) - Adriano Rodrigues.
int led = 13, botao = 12;// Porta do LED - Porta do botao.
void setup(){
pinMode(led, OUTPUT); pinMode(botao, INPUT);// Configurar PDs. }
void loop() {
if (digitalRead(botao) == HIGH) digitalWrite(led, HIGH);
else digitalWrite(led, LOW); }	\end{lstlisting}
\end{frame}
\begin{frame}
	\frametitle{Exercício 2 - Botão de Comando}
	\begin{center}
		\pgfimage[width=\textwidth]{figs/figEx02}
	\end{center}
\end{frame}

%---------------------------------------------- Introdução ao Proteus
\section{Introdução ao Proteus}
\begin{frame}
	\frametitle{Introdução ao Proteus}
	\begin{center}
		\pgfimage[height=50pt]{figs/figProtIcon} \hspace{50pt}
		\pgfimage[height=50pt]{figs/figIsis}
	\end{center}
	Programa que auxilia na simulação de circuitos e componentes elétricos integrados.\\[5pt]
	\begin{itemize}
	\item Simular o Arduino no esquemático do Proteus Isis.
	\end{itemize}
	\begin{center}
		\pgfimage[height=50pt]{figs/figIsisAres}
	\end{center}
\end{frame}

\begin{frame}
	\frametitle{Introdução ao Proteus}
	\begin{itemize}
	\item Inserir Componentes:
		\begin{itemize}
		\item (Barra de Tarefas) Library $>$ Pick parts from libraries.
		\item (Ícones) Amplificador Operacional Dourado $>$ P.\\
		\pgfimage[height=170pt]{figs/figComponet} \hspace{50pt}
		\pgfimage[height=90pt]{figs/figPlace}
		\item (Atalho de teclado) P.
		\end{itemize}
	\end{itemize}
\end{frame}

\begin{frame}
	\frametitle{Introdução ao Proteus}
	\begin{itemize}
	\item Inserir Componentes:
		\begin{itemize}
		\item Arduino (ARDUINO UNO R3);
		\end{itemize}
	\end{itemize}
	\begin{center}
		\pgfimage[width=0.85\textwidth]{figs/figPlaceArduino} \hspace{50pt}
	\end{center}
\end{frame}

\begin{frame}
	\frametitle{Introdução ao Proteus}
	\begin{itemize}
	\item Inserir Componentes:
		\begin{itemize}
		\item Arduino (ARDUINO UNO R3);
		\item Resistor (RES);
		\item Chave (SWITCH);
		\item Led (LED-GREEN, LED-RED, LED-BLUE, LED-YELLOW);
		\item Capacitor (CAP);
		\item Motor DC (MOTOR);
		\item etc...
		\end{itemize}
	\end{itemize}
\end{frame}

\begin{frame}
	\frametitle{Introdução ao Proteus}
	\begin{itemize}
	\item Inserir Alimentação:
		\begin{itemize}
		\item 5V (VCC)
		\item 0V (GND)
		\end{itemize}
	\end{itemize}
		\pgfimage[height=0.70\textheight]{figs/figVccGnd}
\end{frame}

%----------------------------------------------- Exercício 2
\section{Exercício 2 - Proteus}
\begin{frame}
	\frametitle{Exercício 2 - Botão de Comando}
	\begin{center}
		\pgfimage[width=\textwidth]{figs/figEx02}
	\end{center}
\end{frame}

\begin{frame}
	\frametitle{Exercício 2 - Programar Arduino do Proteus}
	Copiar:
	\begin{center}
		\pgfimage[width=0.7\textwidth]{figs/figHex}
	\end{center}
	Clicar duas vezes no Arduino (Proteus), Colar:
	\begin{center}
		\pgfimage[width=0.7\textwidth]{figs/figPasteHex}
	\end{center}
\end{frame}

\begin{frame}
	\frametitle{Exercício 2 - Programar Arduino do Proteus}
	Play:
	\begin{center}
		\pgfimage[width=0.7\textwidth]{figs/figPlay}
	\end{center}
\end{frame}

%---------------------------------------------- Tarefa 1
\section{Tarefa 1}
\begin{frame}[fragile]
	\frametitle{Tarefa 1 - Semáforo no Proteus}
	\begin{itemize}
	\item Criar um código que acenda três LEDs que simulem o funcionamento de um semáforo;
	\item Verificar o funcionamento do código no Proteus.
	\end{itemize}
\end{frame}

\begin{frame}[fragile]
	\frametitle{Tarefa 1 - Semáforo no Proteus}
	\begin{lstlisting}[style=Arduino,basicstyle=\scriptsize \ttfamily]
// Semaforo - Proteus 8 - Adriano Rodrigues.
int LEDg = 11, LEDy = 12, LEDr = 13;
void setup() {
pinMode(LEDg,OUTPUT); pinMode(LEDy,OUTPUT); pinMode(LEDr,OUTPUT);
// Inicia os tres LEDs apagados.
digitalWrite(LEDg,LOW); digitalWrite(LEDy,LOW); digitalWrite(LEDr,LOW); }
void loop() {
// Acende o verde por 5s, depois apaga.
digitalWrite(LEDg,HIGH); delay(5000); digitalWrite(LEDg,LOW);
// Acende o amarelo por 1s, depois apaga.
digitalWrite(LEDy,HIGH); delay(1000); digitalWrite(LEDy,LOW);
// Acende o vermelho por 5s, depois apaga.
digitalWrite(LEDr,HIGH); delay(5000); digitalWrite(LEDr,LOW); } \end{lstlisting}
\end{frame}

%------------------------------------------------ Exercício 3
\section{Exercício 3}
\begin{frame}[fragile]
	\frametitle{Exercício 3 - Frequência Modificada}
	\begin{lstlisting}[style=Arduino,basicstyle=\scriptsize \ttfamily]
// LED e laco for - Adriano Rodrigues.

// Porta do LED.
int led = 13;

void setup(){
// Configuracao da porta digital do LED.
pinMode(led, OUTPUT); }

void loop() {
for (int i = 0;  i < 5; i++) {
digitalWrite(led, HIGH); // Escreve 5V na porta LED.
delay(100*i)           ; // Espera por 100*i milissegundos.
digitalWrite(led, LOW) ; // Escreve 0V na porta LED.
delay(100*i)           ; // Espera por 100*i milissegundos. } } \end{lstlisting}
\end{frame}

%--------------------------------------------- Tarefa 2
\section{Tarefa 2}
\begin{frame}[fragile]
	\frametitle{Tarefa 2 - Frequência Modificada}
	Aumentar o tempo em que cada frequência é apresentada.\\[10pt]
	Cada frequência estava sendo exibida por apenas um período. Propõe-se que cada frequência seja mostrada por uma quantidade maior de períodos, como por exemplo 5 períodos.
\end{frame}

\begin{frame}[fragile]
	\frametitle{Tarefa 2 - Frequência Modificada}
	\begin{lstlisting}[style=Arduino,basicstyle=\scriptsize \ttfamily]
// LED e laco for - Adriano Rodrigues.

void loop() {
for (int i = 0;  i < 5; i++) {
for (int j = 0;  j < 5; j++) {
digitalWrite(led, HIGH); // Escreve 5V na porta LED.
delay(100*i)           ; // Espera por 100*i milissegundos.
digitalWrite(led, LOW) ; // Escreve 0V na porta LED.
delay(100*i)           ; // Espera por 100*i milissegundos. } } } \end{lstlisting}
\end{frame}

%-------------------------------------------------- Sugestão 1
\section{Sugestão 1}
\begin{frame}[fragile]
	\frametitle{Sugestão 1 - 2x Semáforos de um cruzamento}
	\begin{itemize}
	\item Criar um código simule a operação de dois semáforos perpendiculares de um cruzamento;
	\item Verificar o funcionamento no Proteus.
	\end{itemize}
\end{frame}

\section{Fim!}
%----------------------------------- Discussion
\begin{frame}
	\frametitle{Até a próxima aula!}
	\titlepage
\end{frame}

%\begin{frame}[fragile]
%\frametitle{Extras}
%\begin{lstlisting}[style=Arduino]
%// Relogio, Modo menos eficiente, Com delay - Adriano Rodrigues
%float segundos = 0;
%int minutos = 0;
%void setup()
%{
%Serial.begin(9600);
%}
%void loop()
%{
%segundos = millis()-minutos*60000; // Desconta os minutos passados
%segundos = round(segundos/1000); // Arredonda a divisao
%Serial.print(minutos); // Impressao dos minutos e segundos
%Serial.print(':');
%Serial.println(segundos,0);
%if(segundos==59) // Quando chegar em 59, incrementar os minutos
%{
%minutos++;
%}
%delay(1000);
%}
%\end{lstlisting}
%\end{frame}
%
%\begin{frame}[fragile]
%\frametitle{Extras}
%\begin{lstlisting}[style=Arduino,basicstyle=\scriptsize \ttfamily]
%// Relogio, Modo mais eficiente, Sem delay - Adriano Rodrigues
%int segundos = 0, minutos = 0;
%unsigned long timer = 0; // Nao contabilizar negativos
%void setup()
%{
%Serial.begin(9600);
%}
%void loop()
%{
%if (millis()-timer>1000UL) // A cada um segundo (diferenca entre millis e timer), executa:
%{
%timer = millis(); // Faz timer acompanhar o millis.
%Serial.print(minutos);Serial.print(':');Serial.println(segundos);
%segundos++;
%if(segundos == 60) // Zera os segundos e incrementa os minutos.
%{
%segundos = 0;
%minutos++;
%}
%}
%}
%\end{lstlisting}
%\end{frame}
	
\end{document}